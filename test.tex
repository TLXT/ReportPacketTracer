\documentclass[a4paper,12pt]{article}


% --- Cấu hình gói ngôn ngữ và font ---
\usepackage[utf8]{inputenc}
\usepackage[T5]{fontenc}
\usepackage[vietnamese]{babel}

\usepackage{mathptmx}
\usepackage{tabularx}
\usepackage{amssymb}

% --- Cấu hình lề ---
\usepackage[left=2.5cm, right=2cm, top=2cm, bottom=2cm]{geometry}

% --- Gói hình ảnh, màu sắc, bảng ---
\usepackage{graphicx}
\usepackage{float}
\usepackage{xcolor}
\usepackage{multirow}
\usepackage{array}

% --- Gói hiển thị code ---
\usepackage{listings}
\definecolor{codegreen}{rgb}{0,0.6,0}
\definecolor{codegray}{rgb}{0.5,0.5,0.5}
\definecolor{codepurple}{rgb}{0.58,0,0.82}
\definecolor{backcolour}{rgb}{0.95,0.95,0.92}

\usepackage[font=small,labelfont=bf,labelsep=colon]{caption}

\lstdefinestyle{mystyle}{
    backgroundcolor=\color{backcolour},   
    commentstyle=\color{codegreen},
    keywordstyle=\color{magenta},
    numberstyle=\tiny\color{codegray},
    stringstyle=\color{codepurple},
    basicstyle=\ttfamily\footnotesize,
    breakatwhitespace=false,         
    breaklines=true,                 
    captionpos=b,                    
    keepspaces=true,                 
    numbers=left,                    
    numbersep=5pt,                  
    showspaces=false,                
    showstringspaces=false,
    showtabs=false,                  
    tabsize=2
}
\lstset{style=mystyle}

% --- Gói liên kết ---
\usepackage{hyperref}
\hypersetup{
    colorlinks=true,
    linkcolor=black,
    filecolor=magenta,      
    urlcolor=blue,
    pdftitle={Bao Cao Do An Mang May Tinh},
}

\usepackage{fancyhdr}
\pagestyle{fancy}
\fancyhf{}
\lhead{\small \textbf{Báo cáo Đồ Án Môn Học}}
\rhead {\small \textit{Trường Đại học Khoa học tự nhiên, ĐHQG-HCM - Khoa CNTT}}
\lfoot{\small \textit{By: 24120136 - 24120157 - 24120394}}
\rfoot{\small \thepage}
\renewcommand{\headrulewidth}{0.4pt}
\renewcommand{\footrulewidth}{0.4pt}


\usepackage{titlesec}
% Mục lớn (Section): Chữ to, in đậm, màu xanh dương đậm
\titleformat{\section}
  {\normalfont\Large\bfseries\color{blue!70!black}}{\thesection}{1em}{}
% Mục nhỏ (Subsection): In đậm, màu đen
\titleformat{\subsection}
  {\normalfont\large\bfseries}{\thesubsection}{1em}{}
% ===================================================
% BẮT ĐẦU VĂN BẢN
% ===================================================
\begin{document}

% --- CHÈN ĐOẠN ĐỔI TÊN TẠI ĐÂY (SAU \begin{document}) ---
% Cách này sẽ ép buộc LaTeX đổi tên ngay lập tức
\renewcommand{\contentsname}{Mục lục}
\renewcommand{\listfigurename}{Danh mục hình ảnh}
\renewcommand{\listtablename}{Danh mục bảng biểu}
\renewcommand{\figurename}{Hình}
\renewcommand{\tablename}{Bảng}
\renewcommand{\refname}{Tài liệu tham khảo}
% --------------------------------------------------------

% ====================================================================
% TRANG BÌA
% ====================================================================
\begin{titlepage}
    \centering
    \textbf{\large ĐẠI HỌC QUỐC GIA TP.HCM}\\
    \textbf{\large TRƯỜNG ĐẠI HỌC KHOA HỌC TỰ NHIÊN}\\
    \textbf{\large KHOA CÔNG NGHỆ THÔNG TIN}\\
    \textbf{\large NHÓM NGÀNH MÁY TÍNH VÀ CÔNG NGHỆ THÔNG TIN}
    
    \vspace{2cm}
    
    \includegraphics[width=0.4\textwidth]{logo_khtn.png} % Thay 'logo_khtn.png' bằng tên file ảnh logo của bạn
    
    \vspace{2cm}
    
    {\bfseries \LARGE BÁO CÁO ĐỒ ÁN MÔN HỌC \par}
    \vspace{0.5cm}
    {\bfseries \Large MÔN: MẠNG MÁY TÍNH \par}
    \vspace{0.5cm}
    {\bfseries \Large ĐỀ TÀI: PACKET TRACER \par}
    
    \vspace{2cm}

\vspace{1cm} % Khoảng cách giữa tiêu đề và danh sách

\noindent % Không thụt đầu dòng
\begin{minipage}[t]{0.6\textwidth} % Cột trái chiếm 60% chiều rộng
    \textit{Sinh viên thực hiện:} \\[6pt] % In nghiêng dòng tiêu đề
    Nguyễn Đặng Khôi Nguyên (24120394) \\[4pt]
    Lê Thái Vinh (24120157) \\[4pt]
    Trần Lê Xuân Tân (24120136)
\end{minipage}%
\hfill % Đẩy cột phải sang phía bên phải
\begin{minipage}[t]{0.35\textwidth} % Cột phải chiếm 35% chiều rộng
    \raggedleft % Căn lề phải cho nội dung trong cột này
    \textit{Giáo viên hướng dẫn:} \\[6pt]
    Lê Hà Minh
\end{minipage}
    
    

    \vfill
    
    {\large Tp. Hồ Chí Minh, tháng 12 năm 2025}
\end{titlepage}

% ====================================================================
% MỤC LỤC
% ====================================================================
\newpage
\tableofcontents
\newpage
\listoffigures % Tạo danh sách hình ảnh
\newpage

% ====================================================================
% NỘI DUNG CHÍNH
% ====================================================================

\section{THÔNG TIN NHÓM}

\subsection{Giới thiệu thành viên nhóm}

\begin{table}[h]
        \centering
        \begin{tabular}{|c|l|c|l|c|l}
        \hline
        \textbf{MSSV} & \textbf{Họ và Tên} & \textbf{Địa chỉ email} & \textbf{Ghi chú} \\
        \hline
        24120394 & Nguyễn Đặng Khôi Nguyên & 24120394@student.hcmus.edu.vn & \\
        \hline
        24120136 & Trần Lê Xuân Tân & 24120136@student.hcmus.edu.vn & \\
        \hline
        23120157 & Lê Thái Vinh & 24120157@student.hcmus.edu.vn & \\
        \hline
        \end{tabular}
    \end{table}

\subsection{Mục tiêu đồ án}
\begin{itemize}
    \item Thiết kế và triển khai hệ thống mạng đa tầng cho doanh nghiệp sử dụng Cisco Packet Tracer.
    \item Thành thạo cấu hình định tuyến tĩnh (Static Route) và định tuyến động (RIPv2).
    \item Triển khai các dịch vụ mạng thiết yếu: DHCP Server, DNS Server và Web Server.
    \item Phân tích chi tiết hành vi gói tin qua các tầng OSI trong môi trường giả lập.
\end{itemize}

\section{ĐÁNH GIÁ MỨC ĐỘ HOÀN THÀNH}

\subsection{Bảng đánh giá mức độ hoàn thành}
\begin{table}[h]
    \centering
    \begin{tabular}{|c|l|c|}
    \hline
    \textbf{STT} & \textbf{Nội dung thực hiện} & \textbf{Mức độ hoàn thành} \\ \hline
    1 & Thiết kế Topology và cấu hình IP Part 1 & 100\% \\ \hline
    2 & Cấu hình Định tuyến tĩnh (Static Routing) & 100\% \\ \hline
    3 & Chia mạng con và cấu hình DHCP Relay Agent & 100\% \\ \hline
    4 & Triển khai DNS và Web Server & 100\% \\ \hline
    5 & Định tuyến động RIPv2 thông suốt toàn mạng & 100\% \\ \hline
    \end{tabular}
\end{table}

\subsection{Số điểm mong đợi}
Nhóm mong muốn đạt điểm tuyệt đối cho đồ án này dựa trên việc hoàn thành tất cả các yêu cầu kỹ thuật và phân tích gói tin chi tiết.

% --- PART 1 ---
\section{PART 1: STATIC ROUTING AND PACKET ANALYSIS}

\subsection{Phần chuẩn bị (Cấu hình hệ thống)}
Trong phần này, nhóm đã cấu hình IP tĩnh cho các thiết bị đầu cuối và thiết lập các câu lệnh định tuyến tĩnh trên các Router.
\begin{itemize}
    \item \textbf{Cấu hình IP:} PC1 ($192.168.1.10$), PC2 ($192.168.2.10$), PC3 ($192.168.3.10$).
    \item \textbf{Định tuyến tĩnh:} Thực hiện lệnh \texttt{ip route} trên các Router để chỉ đường đến các mạng LAN xa.
\end{itemize}

\textbf{Các bước chi tiết:}

\begin{figure}[H]
    \centering
    \includegraphics[width=1\textwidth]{B1.png} 
    \caption{Chuẩn bị 3 Router và đặt tên lần lượt là R1, R2, R3}
\end{figure}

\begin{figure}[H]
    \centering
    \includegraphics[width=1\textwidth]{B2.png} 
    \caption{Chuẩn bị thêm 3 Switch và đặt tên lần lượt là S1, S2, S3}
\end{figure}

\begin{figure}[H]
    \centering
    \includegraphics[width=1\textwidth]{B3.png} 
    \caption{Chuẩn bị thêm 3 PC và đặt tên lần lượt là PC1, PC2, PC3}
\end{figure}

\begin{figure}[H]
    \centering
    \includegraphics[width=1\textwidth]{B4.png} 
    \caption{Chọn R1 vào Physical chọn HWIC-2T và tắt thiết bị}
\end{figure}

\begin{figure}[H]
    \centering
    \includegraphics[width=1\textwidth]{B5.png} 
    \caption{Lắp HWIC-2T vào thiết bị để có thêm 2 cổng Serial High-Speed}
\end{figure}

\begin{figure}[H]
    \centering
    \includegraphics[width=1\textwidth]{B6.png} 
    \caption{Turn On Router và chờ vài giây}
\end{figure}

Tương tự 3 bước trên để lắp HWIC-2T cho R2, R3.

\begin{figure}[H]
    \centering
    \includegraphics[width=1\textwidth]{B7.png} 
    \caption{Bấm biểu tường hình sấm sét để chọn dây đỏ có đồng hồ}
\end{figure}

\begin{figure}[H]
    \centering
    \includegraphics[width=1\textwidth]{B8.png} 
    \caption{Bấm vào R1 và chọn cổng Serial0/0/0}
\end{figure}

\begin{figure}[H]
    \centering
    \includegraphics[width=1\textwidth]{B9.png} 
    \caption{Bấm tiếp vào R2 và chọn cổng Serial0/0/0 để nối dây}
\end{figure}

\begin{figure}[H]
    \centering
    \includegraphics[width=1\textwidth]{B10.png} 
    \caption{Tương tự 2 bước trên nối cổng Serial0/0/1 của R2 với cổng Serial0/0/1 của R3}
\end{figure}

\begin{figure}[H]
    \centering
    \includegraphics[width=1\textwidth]{B11.png} 
    \caption{Hình ảnh sau khi nối các Router với nhau}
\end{figure}

\begin{figure}[H]
    \centering
    \includegraphics[width=1\textwidth]{B12.png} 
    \caption{Bấm vào R1 và chọn cổng GigabitEthernet0/0}
\end{figure}

\begin{figure}[H]
    \centering
    \includegraphics[width=1\textwidth]{B13.png} 
    \caption{Bấm tiếp vào S1 và chọn cổng GigabitEthernet0/1 để nối R1 và S1}
\end{figure}

\begin{figure}[H]
    \centering
    \includegraphics[width=1\textwidth]{B14.png} 
    \caption{Bấm vào S1 và chọn cổng FastEthernet0/2}
\end{figure}

\begin{figure}[H]
    \centering
    \includegraphics[width=1\textwidth]{B15.png} 
    \caption{Bấm tiếp vào PC1 và chọn cổng FastEthernet0 để nối S1 và PC1}
\end{figure}

Làm tương tự các bước trên để nối R2-S2-PC2 và R3-S3-PC3.

\begin{figure}[H]
    \centering
    \includegraphics[width=1\textwidth]{B16.png} 
    \caption{Hình ảnh sau khi nối R2-S2-PC2 và R3-S3-PC3}
\end{figure}

\begin{figure}[H]
    \centering
    \includegraphics[width=1\textwidth]{B17.png} 
    \caption{Bấm vào R1 chọn CLI}
\end{figure}

\begin{figure}[H]
    \centering
    \includegraphics[width=1\textwidth]{B18.png} 
    \caption{Nhập các dòng lệnh này vào CLI của R1}
\end{figure}

\begin{figure}[H]
    \centering
    \includegraphics[width=1\textwidth]{B19.png}
    \includegraphics[width=1\textwidth]{B19.5.png}
    \caption{Tương tự mở CLI của R2 và nhập các dòng lệnh này vào CLI của R2}
\end{figure}

\begin{figure}[H]
    \centering
    \includegraphics[width=1\textwidth]{B20.png}
    \caption{Tương tự mở CLI của R3 và nhập các dòng lệnh này vào CLI của R3}
\end{figure}

Sau khi nhập các dòng lệnh trên thì chúng ta sẽ thấy đường dây nối giữa các Router sẽ đổi từ các mũi tên màu đỏ thành xanh lá.

\begin{figure}[H]
    \centering
    \includegraphics[width=1\textwidth]{B21.png}
    \caption{Quay lại CLI của R1 để nhập thêm các dòng lệnh trên vào}
\end{figure}

\begin{figure}[H]
    \centering
    \includegraphics[width=1\textwidth]{B22.png}
    \caption{Quay lại CLI của R2 để nhập thêm các dòng lệnh trên vào}
\end{figure}

\begin{figure}[H]
    \centering
    \includegraphics[width=1\textwidth]{B23.png}
    \caption{Quay lại CLI của R3 để nhập thêm các dòng lệnh trên vào}
\end{figure}

\begin{figure}[H]
    \centering
    \includegraphics[width=1\textwidth]{B24.png}
    \caption{Hình ảnh sau khi đã nhập hết tất cả các lệnh trên}
\end{figure}

\begin{figure}[H]
    \centering
    \includegraphics[width=1\textwidth]{B25.png}
    \caption{Bấm vào PC1 chọn Desktop -> IP Configuration và sửa các IP của PC1 theo yêu cầu đề bài}
\end{figure}

\begin{figure}[H]
    \centering
    \includegraphics[width=1\textwidth]{B26.png}
    \caption{Tương tự sửa các địa chỉ IP của PC2 theo yêu cầu đề bài}
\end{figure}

\begin{figure}[H]
    \centering
    \includegraphics[width=1\textwidth]{B27.png}
    \caption{Tương tự sửa các địa chỉ IP của PC3 theo yêu cầu đề bài}
\end{figure}

\textbf{Kiểm tra:}
\begin{itemize}
    \item Từ PC1, mở Command Prompt và ping đến địa chỉ IP của PC3 (192.168.3.10) để kiểm tra kết nối. Ta sẽ thấy kết quả trả về thành công.
    \begin{figure}[H]
        \centering
        \includegraphics[width=1\textwidth]{C28.png}
        \caption{Ping từ PC1 đến PC3 thành công}
    \end{figure}
    \item Từ Router R1, sử dụng lệnh \texttt{show ip route} để kiểm tra bảng định tuyến. Ta sẽ thấy IP của Router R2 và R3 đã xuất hiện trong bảng định tuyến và được đánh dấu là định tuyến tĩnh (S).
    \begin{figure}[H]
        \centering
        \includegraphics[width=1\textwidth]{C29.png}
        \caption{Kiểm tra bảng định tuyến trên R1}
    \end{figure}
    \item Từ Router R2, sử dụng lệnh \texttt{show ip route} để kiểm tra bảng định tuyến. Ta sẽ thấy IP của Router R1 và R3 đã xuất hiện trong bảng định tuyến và được đánh dấu là định tuyến tĩnh (S).
    \begin{figure}[H]
        \centering
        \includegraphics[width=1\textwidth]{C30.png}
        \caption{Kiểm tra bảng định tuyến trên R2}
    \end{figure}
    \item Từ Router R3, sử dụng lệnh \texttt{show ip route} để kiểm tra bảng định tuyến. Ta sẽ thấy IP của Router R1 và R2 đã xuất hiện trong bảng định tuyến và được đánh dấu là định tuyến tĩnh (S).
    \begin{figure}[H]
        \centering
        \includegraphics[width=1\textwidth]{C31.png}
        \caption{Kiểm tra bảng định tuyến trên R3}
    \end{figure}
\end{itemize}
Vậy là chúng ta đã hoàn thành xong phần cấu hình hệ thống cho Part 1. Tiếp theo chúng ta sẽ phân tích hành trình gói tin từ PC1 đến PC3.
\newpage


\subsection{Trả lời các câu hỏi Lab - Phân tích hành trình gói tin (PC1 $\rightarrow$ PC3)}

\subsubsection*{Phase 1: Gói tin rời LAN nguồn (PC1 $\rightarrow$ R1)}

\begin{figure}[H]
    \centering
    \includegraphics[width=1\textwidth]{B31.png}
    \caption{Gói tin ở PC1 trước khi rời đi}
\end{figure}

\begin{figure}[H]
    \centering
    \includegraphics[width=1\textwidth]{B32.png}
    \caption{Gói tin khi đến S1}
\end{figure}

\begin{figure}[H]
    \centering
    \includegraphics[width=1\textwidth]{B33.png}
    \caption{Gói tin khi đến R1}
\end{figure}

\begin{itemize}
    \item \textbf{Câu hỏi:} Khi gói tin vừa được tạo trên PC1, địa chỉ IP nguồn và IP đích là gì? Những địa chỉ này có thay đổi trong suốt hành trình không?
    \item \textbf{Trả lời:} Địa chỉ IP nguồn là $192.168.1.10$ và địa chỉ IP đích là $192.168.3.10$. Các địa chỉ này \textbf{không bao giờ thay đổi} trong suốt hành trình truyền tải vì chúng dùng để định danh thiết bị cuối (End-to-End).

    \item \textbf{Câu hỏi:} Làm thế nào PC1 xác định được địa chỉ MAC đích cần thiết để gửi gói tin đến chặng kế tiếp (R1)? Giao thức nào được sử dụng?
    \item \textbf{Trả lời:} PC1 nhận thấy IP đích nằm ngoài mạng LAN nên nó gửi gói tin đến Default Gateway (R1). PC1 sử dụng giao thức \textbf{ARP (Address Resolution Protocol)} để phân giải địa chỉ IP của cổng Gateway thành địa chỉ MAC tương ứng.

    \item \textbf{Câu hỏi:} Khi gói tin rời PC1 và đi qua Switch (S1), địa chỉ MAC nguồn và MAC đích của khung Ethernet là gì?
    \item \textbf{Trả lời:} Địa chỉ MAC nguồn là địa chỉ vật lý của PC1 và địa chỉ MAC đích là địa chỉ vật lý của cổng GigabitEthernet0/0 trên Router R1.
\end{itemize}

\subsubsection*{Phase 2: Trải qua lộ trình định tuyến (R1 $\rightarrow$ R2 $\rightarrow$ R3)}

\begin{figure}[H]
    \centering
    \includegraphics[width=1\textwidth]{B34.png}
    \caption{Gói tin khi đến R2}
\end{figure}

\begin{figure}[H]
    \centering
    \includegraphics[width=1\textwidth]{B35.png}
    \caption{Gói tin khi đến R3}
\end{figure}

\begin{itemize}
    \item \textbf{Câu hỏi:} Khi khung dữ liệu đến R1, điều gì xảy ra với địa chỉ MAC đích? Tại sao Router lại hủy bỏ tiêu đề khung ban đầu? Địa chỉ Next Hop IP để đến $192.168.3.10$ là gì?
    \item \textbf{Trả lời:} R1 sẽ \textbf{gỡ bỏ (discard)} tiêu đề khung Ethernet cũ vì thông tin Lớp 2 chỉ có giá trị nội bộ trong một chặng đơn lẻ. Router cần đọc gói tin ở Lớp 3 để thực hiện định tuyến. Địa chỉ \textbf{Next Hop IP} để đi tiếp là $10.0.12.2$ (cổng Serial của R2).

    \item \textbf{Câu hỏi:} Liên kết R1 đến R2 (Serial): Địa chỉ MAC nguồn và MAC đích trong khung mới do R1 tạo ra là gì?
    \item \textbf{Trả lời:} Vì đây là liên kết Serial điểm-đến-điểm, nó không sử dụng địa chỉ MAC như Ethernet. R1 đóng gói gói tin bằng tiêu đề Serial (như HDLC hoặc PPP) thay thế cho Ethernet frame.

    \item \textbf{Câu hỏi:} Khi gói tin đến R2, Router này có ghi lại địa chỉ IP nguồn/đích không? Giải thích theo quy tắc Lớp 3.
    \item \textbf{Trả lời:} R2 \textbf{không} ghi lại IP nguồn hay IP đích. Theo quy tắc Layer 3, địa chỉ IP phải được giữ nguyên từ nguồn đến đích để đảm bảo gói tin đến đúng nơi và thiết bị đích có thể phản hồi lại đúng địa chỉ nguồn.

    \item \textbf{Câu hỏi:} R2 làm gì với tiêu đề Lớp 2 trước khi gửi đến R3? Khung mới của R2 so với R1 như thế nào?
    \item \textbf{Trả lời:} R2 gỡ bỏ tiêu đề Lớp 2 nhận được từ R1 và tạo ra một tiêu đề Lớp 2 mới cho chặng R2-R3. Khung mới này có cấu trúc tương tự khung của R1 vì cả hai chặng đều sử dụng công nghệ truyền dẫn Serial.
\end{itemize}

\subsubsection*{Phase 3: Gói tin đến LAN đích (R3 $\rightarrow$ PC3)}

\begin{figure}[H]
    \centering
    \includegraphics[width=1\textwidth]{B36.png}
    \caption{Gói tin khi đến S3}
\end{figure}

\begin{figure}[H]
    \centering
    \includegraphics[width=1\textwidth]{B37.png}
    \caption{Gói tin khi đến PC3}
\end{figure}

\begin{itemize}
    \item \textbf{Câu hỏi:} Khi gói tin đến R3, giao tiếp nào được xác định là cổng thoát cho địa chỉ $192.168.3.10$?
    \item \textbf{Trả lời:} R3 xác định cổng thoát là \textbf{GigabitEthernet0/0}, giao tiếp kết nối trực tiếp với mạng LAN của PC3.

    \item \textbf{Câu hỏi:} Trước khi gửi đến PC3, R3 phải khám phá địa chỉ nào và bằng cách nào?
    \item \textbf{Trả lời:} R3 phải khám phá địa chỉ \textbf{MAC vật lý của PC3}. Nó thực hiện việc này bằng cách gửi một gói tin \textbf{ARP Request} xuống mạng LAN.

    \item \textbf{Câu hỏi:} Địa chỉ MAC nguồn và MAC đích khi khung dữ liệu rời cổng LAN của R3 đến PC3 là gì?
    \item \textbf{Trả lời:} Địa chỉ MAC nguồn là địa chỉ vật lý cổng Gig0/0 của Router R3 và địa chỉ MAC đích là địa chỉ vật lý của PC3.
\end{itemize}

Sau khi phân tích hành trình gói tin từ PC1 đến PC3, chúng ta thấy rằng địa chỉ IP nguồn và đích không thay đổi trong suốt quá trình truyền. Tuy nhiên, địa chỉ MAC thay đổi tại mỗi chặng để phù hợp với các liên kết vật lý khác nhau. Các Router thực hiện việc tra cứu bảng định tuyến để xác định Next Hop và đóng gói lại gói tin theo giao thức phù hợp với từng chặng đường.

Và ngoài ra, nhóm cũng đã kiểm tra kết nối ngược lại từ PC3 đến PC1 và kết quả cũng tương tự như trên, đảm bảo tính hai chiều trong việc truyền dữ liệu.

\begin{figure}[H]
    \centering
    \includegraphics[width=1\textwidth]{B38.png}
    \caption{Gói tin đi từ PC3 đến S3}
\end{figure}

\begin{figure}[H]
    \centering
    \includegraphics[width=1\textwidth]{B39.png}
    \caption{Gói tin đến R3}
\end{figure}

\begin{figure}[H]
    \centering
    \includegraphics[width=1\textwidth]{B40.png}
    \caption{Gói tin đến R2}
\end{figure}

\begin{figure}[H]
    \centering
    \includegraphics[width=1\textwidth]{B41.png}
    \caption{Gói tin đến R1}
\end{figure}

\begin{figure}[H]
    \centering
    \includegraphics[width=1\textwidth]{B42.png}
    \caption{Gói tin đến S1}
\end{figure}

\begin{figure}[H]
    \centering
    \includegraphics[width=1\textwidth]{B43.png}
    \caption{Gói tin đến PC1}
\end{figure}

Ta thấy rằng hành trình gói tin từ PC3 đến PC1 cũng tuân theo các nguyên tắc tương tự như từ PC1 đến PC3, với việc thay đổi địa chỉ MAC tại mỗi chặng và giữ nguyên địa chỉ IP nguồn và đích.

\newpage

\section{PART 2: ENTERPRISE SERVICES AND DHCP RELAY}

\subsection{Thiết kế và Phân hoạch địa chỉ IP}
\subsubsection{Yêu cầu thiết kế}
\begin{itemize}
    \item \textbf{Mục tiêu:} Mở rộng hệ thống mạng doanh nghiệp, kết nối chi nhánh mới (Part 2) vào hệ thống trụ sở chính (Part 1).
    \item \textbf{Yêu cầu kỹ thuật:} 
    \begin{itemize}
        \item \textbf{Office LAN:} 30 hosts (Dành cho nhân viên).
        \item \textbf{Server Farm:} 10 hosts (Dành cho các máy chủ dịch vụ).
        \item \textbf{WAN Link:} 2 hosts (Kết nối Router R2 - R4).
    \end{itemize}
    \item \textbf{Địa chỉ mạng gốc:} $172.16.0.0/16$.
\end{itemize}
\subsubsection{Bảng phân hoạch IP (VLSM Calculation)}
\begin{table}[h!]
    \centering
    \renewcommand{\arraystretch}{1.4}
    \begin{tabular}{|l|c|c|c|c|c|p{3.5cm}|}
        \hline
        \shortstack{\textbf{Tên mạng}\\\textbf{(Network Name)}} &
        \textbf{VLAN ID} &
        \shortstack{\textbf{Số Host}\\\textbf{yêu cầu}} &
        \textbf{CIDR} &
        \textbf{Subnet Mask} &
        \shortstack{\textbf{Địa chỉ mạng}\\\textbf{(Network Address)}} \\
        \hline
        WAN Link &
        N/A &
        2 &
        /30 &
        255.255.255.252 &
        172.16.0.48 \\
        \hline
        Office LAN &
        10 &
        30 &
        /27 &
        255.255.255.224 &
        172.16.0.0 \\
        \hline
        Server Farm &
        30 &
        10 &
        /28 &
        255.255.255.240 &
        172.16.0.32 \\
        \hline
    \end{tabular}
\end{table}

\subsection{\textbf{Giai đoạn 1: Triển khai Kết nối cơ bản (Mandatory - No VLANs)}}
\subsubsection{\textbf{Sơ đồ đấu nối thiết bị}}
\textbf{Mô tả:} Nhóm thiết lập mô hình kết nối vật lý như sau:
\begin{itemize}
    \item \textbf{Router R4:} Đóng vai trò Gateway cho toàn bộ chi nhánh mới.
    \item \textbf{Cổng G0/0:} Kết nối xuống Switch S4 (Office LAN).
    \item \textbf{Cổng G0/1:} Kết nối xuống Switch S5 (Server Farm).
    \item \textbf{Cổng Serial0/0/0:} Kết nối WAN về Router R2 (Part 1).
\end{itemize}


\textbf{Các bước chi tiết:}
\begin{figure}[H]
    \centering
    \includegraphics[width=1\textwidth]{P2_1.png}
    \caption{Chuẩn bị Router R4, Switch S4, S5 và các PC, Server}
\end{figure}
\begin{figure}[H]
    \centering
    \includegraphics[width=1\textwidth]{P2_2.png}
    \caption{Lắp HWIC-2T cho R4 để có thêm 2 cổng Serial}
\end{figure}
\begin{figure}[H]
    \centering
    \includegraphics[width=1\textwidth]{P2_3.png}
    \caption{Lắp HWIC-2T thứ 2 cho R2 để có thêm 2 cổng Serial}
\end{figure}
\begin{figure}[H]
    \centering
    \includegraphics[width=1\textwidth]{P2_4.png}
    \caption{Chọn dây DCE, bấm vào R2 và chọn cổng Serial0/1/0}
\end{figure}
\begin{figure}[H]
    \centering
    \includegraphics[width=1\textwidth]{P2_5.png}
    \caption{Bấm tiếp vào R4 và chọn cổng Serial0/0/0 để nối dây}
\end{figure}
\begin{figure}[H]
    \centering
    \includegraphics[width=1\textwidth]{P2_6.png}
    \caption{Chọn dây Copper Straight-Through, bấm vào R4 và chọn cổng GigabitEthernet0/0}
\end{figure}
\begin{figure}[H]
    \centering
    \includegraphics[width=1\textwidth]{P2_7.png}
    \caption{Bấm tiếp vào S4 và chọn cổng GigabitEthernet0/1 để nối dây}
\end{figure}
\begin{figure}[H]
    \centering
    \includegraphics[width=1\textwidth]{P2_8.png}
    \caption{Chọn dây Copper Straight-Through, bấm vào S4 và chọn cổng FastEthernet0/1}
\end{figure}
\begin{figure}[H]
    \centering
    \includegraphics[width=1\textwidth]{P2_9.png}
    \caption{Bấm tiếp vào PC4 và chọn cổng FastEthernet0 để nối dây}
\end{figure}
Làm tương tự để nối Router R4 với Switch S5 qua cổng GigabitEthernet0/1 và Switch S4 với các PC còn lại.
\begin{figure}[H]
    \centering
    \includegraphics[width=1\textwidth]{P2_10.png}
    \caption{Chọn dây Copper Straight-Through, bấm vào S5 và chọn cổng FastEthernet0/1}
\end{figure}
\begin{figure}[H]
    \centering
    \includegraphics[width=1\textwidth]{P2_11.png}
    \caption{Bấm tiếp vào ServerDHCP/DNS và chọn cổng FastEthernet0 để nối dây}
\end{figure}
Làm tương tự để nối Switch S5 với ServerHTTP qua cổng FastEthernet0/2.
\newpage
\begin{figure}[H]
    \centering
    \includegraphics[width=1\textwidth]{P2_12.png}
    \caption{Hình ảnh sau khi nối xong tất cả các thiết bị trong Part 2}
\end{figure}
\subsubsection{\textbf{Thiết lập kết nối WAN (WAN Link Setup)}}
\textbf{Mô tả:} Nhóm tiến hành cấu hình địa chỉ IP cho cổng Serial trên Router R4 và Router R2 để thiết lập đường truyền vật lý giữa hai chi nhánh.
\begin{itemize}
    \item \textbf{Trên Router R4:} Cấu hình cổng Serial0/0/0 với IP 172.16.0.50/30 và bật cổng (no shutdown).
    \begin{figure}[H]
        \centering
        \includegraphics[width=1\textwidth]{P2_13.png}
        \caption{Mở CLI của R4 và nhập các dòng lệnh để cấu hình cổng Serial0/0/0}
    \end{figure}
    \item \textbf{Trên Router R2:} Cấu hình cổng Serial0/1/0 với IP 172.16.0.49/30 và thiết lập clock rate 64000 (đầu DCE).
    \begin{figure}[H]
        \centering
        \includegraphics[width=1\textwidth]{P2_14.png}
        \caption{Mở CLI của R2 và nhập các dòng lệnh để cấu hình cổng Serial0/1/0}
    \end{figure}
\end{itemize}
Sau khi hoàn thành, hai Router R2 và R4 đã có thể giao tiếp với nhau qua đường truyền WAN.
\begin{figure}[H]
    \centering
    \includegraphics[width=1\textwidth]{P2_15.png}
    \caption{Kết nối giữa R2 và R4 chuyển xanh}
\end{figure}
\subsubsection{\textbf{Cấu hình các cổng LAN Gateway (Router LAN Interfaces)}}
\textbf{Mô tả:} Nhóm cấu hình địa chỉ IP cho các cổng GigabitEthernet trên Router R4 để làm Gateway cho các mạng LAN trong chi nhánh mới.
\begin{itemize}
    \item \textbf{Cổng G0/0 (Office LAN):} Cấu hình với IP 172.16.0.1/27.
    \begin{figure}[H]
        \centering
        \includegraphics[width=1\textwidth]{Giga00.png}
        \caption{Mở CLI của R4 và nhập các dòng lệnh để cấu hình cổng GigabitEthernet0/0}
    \end{figure}
    \item \textbf{Cổng G0/1 (Server Farm):} Cấu hình với IP 172.16.0.33/28.
    \begin{figure}[H]
        \centering
        \includegraphics[width=1\textwidth]{Giga01.png}
        \caption{Mở CLI của R4 và nhập các dòng lệnh để cấu hình cổng GigabitEthernet0/1}
    \end{figure}
\end{itemize}
Sau khi hoàn thành, các cổng LAN trên Router R4 đã sẵn sàng làm Gateway cho các thiết bị trong mạng LAN (kết nối chuyển xanh).
\begin{figure}[H]
    \centering
    \includegraphics[width=1\textwidth]{P2_16.png}
    \caption{Kết nối giữa R4 và S4, S5 chuyển xanh}
\end{figure}
\subsubsection{\textbf{Cấu hình IP tĩnh cho Server (Server IP Configuration)}}
\textbf{Mô tả:} Nhóm cấu hình địa chỉ IP tĩnh cho các Server trong Server Farm để đảm bảo chúng luôn có địa chỉ cố định.
\begin{itemize}
    \item \textbf{ServerDHCP/DNS:} Cấu hình với IP 172.16.0.34, Gateway 172.16.0.33.
    \begin{figure}[H]
        \centering
        \includegraphics[width=1\textwidth]{ServerDHCP.png}
        \caption{Cấu hình IP tĩnh cho ServerDHCP/DNS}
    \end{figure}
    \item \textbf{ServerHTTP:} Cấu hình với IP 172.16.0.35, Gateway 172.16.0.33.
    \begin{figure}[H]
        \centering
        \includegraphics[width=1\textwidth]{ServerHTTP.png}
        \caption{Cấu hình IP tĩnh cho ServerHTTP}
    \end{figure}
\end{itemize}
Khi chạy ipconfig trên các Server, ta thấy địa chỉ IP đã được cấu hình đúng.
\begin{figure}[H]
    \centering
    \includegraphics[width=1\textwidth]{ipconfigDHCPDNS.png}
    \caption{Chạy ipconfig trên ServerDHCP/DNS}
\end{figure}
\begin{figure}[H]
    \centering
    \includegraphics[width=1\textwidth]{ipconfigHTTP.png}
    \caption{Chạy ipconfig trên ServerHTTP}
\end{figure}
\subsubsection{\textbf{Cấu hình Định tuyến tĩnh toàn mạng (End-to-End Static Routing)}}
\textbf{Mô tả:} Nhóm cấu hình định tuyến tĩnh trên Router R4 để đảm bảo các mạng LAN trong chi nhánh mới có thể giao tiếp với mạng trụ sở chính.
\begin{itemize} 
    \item \textbf{Trên Router R4:} Cấu hình Default Route trỏ về R2 để ra Internet.
    \begin{figure}[H]
        \centering
        \includegraphics[width=1\textwidth]{P2_17.png}
        \caption{Mở CLI của R4 và nhập các dòng lệnh để cấu hình Default Route}
    \end{figure}
    \item \textbf{Trên Router R2:} Cấu hình Route trỏ về mạng 172.16.0.0/16 qua R4.
    \begin{figure}[H]
        \centering
        \includegraphics[width=1\textwidth]{P2_18.png}
        \caption{Mở CLI của R2 và nhập các dòng lệnh để cấu hình Route về mạng 172.16.0.0/16 qua R4.}
    \end{figure}
    \item \textbf{Trên Router R1 và R3:} Cấu hình Route trỏ về mạng 172.16.0.0/16 thông qua R2 (Next-hop).
    \begin{figure}[H]
        \centering
        \includegraphics[width=1\textwidth]{P2_19.png}
        \caption{Cập nhật bảng định tuyến trên R1 để nhận biết mạng mới.}
    \end{figure}
    \begin{figure}[H]
        \centering
        \includegraphics[width=1\textwidth]{P2_20.png}
        \caption{Cập nhật bảng định tuyến trên R3 để nhận biết mạng mới.}
    \end{figure}
\end{itemize}
Sau khi hoàn thành, toàn bộ mạng trong chi nhánh mới đã có thể giao tiếp với mạng trụ sở chính.
\begin{itemize}
    \item \textbf{Kiểm tra các kết nối của Router R4:}
    \begin{figure}[H]
        \centering
        \includegraphics[width=1\textwidth]{P2_21.png}
        \caption{Chạy lệnh \texttt{show ip route} trên R4 để kiểm tra bảng định tuyến}
    \end{figure}
    \item \textbf{Kiểm tra các kết nối của Router R2:}
    \begin{figure}[H]
        \centering
        \includegraphics[width=1\textwidth]{P2_22.png}
        \caption{Chạy lệnh \texttt{show ip route} trên R2 để kiểm tra bảng định tuyến}
    \end{figure}
\end{itemize}
\subsubsection{\textbf{Triển khai Dịch vụ Mạng trên Server (Core Services Setup)}}
\textbf{Mô tả:} Nhóm triển khai các dịch vụ mạng thiết yếu trên các Server trong chi nhánh mới, tiến hành cấu hình phần mềm dịch vụ.
\begin{itemize}
    \item \textbf{Trên ServerDHCP/DNS:} 
    Chọn mục Service/DHCP trên Server, tạo serverPool với các thông số:
        \begin{itemize}
            \item Pool Name: officePool
            \item Default Gateway: 172.16.0.1
            \item DNS: 172.16.0.34
            \item Starting IP Address: 172.16.0.2
            \item Subnet Mask: 255.255.255.224
            \item Maximum Number of Users: 30
        \end{itemize}
        Ấn nút Add để hoàn tất cấu hình DHCP Server.
    \begin{figure}[H]
        \centering
        \includegraphics[width=1\textwidth]{P2_23.png}
        \caption{Cấu hình DHCP Server trên ServerDHCP/DNS}
    \end{figure}
    Chọn mục Service/DNS trên Server, tạo bản ghi DNS với các thông số:
        \begin{itemize}
            \item Hostname: companyweb.com
            \item Type : A
            \item IP Address: 172.16.0.35
        \end{itemize}
        Ấn nút Add để hoàn tất cấu hình DNS Server.
    \begin{figure}[H]
        \centering 
        \includegraphics[width=1\textwidth]{P2_24.png}
        \caption{Cấu hình DNS Server trên ServerDHCP/DNS}   
    \end{figure}
    \item \textbf{Trên ServerHTTP:}
    Chọn mục Service/HTTP trên Server, bật dịch vụ HTTP và HTTPS.
    \begin{figure}[H]
        \centering
        \includegraphics[width=1\textwidth]{P2_25.png}
        \caption{Cấu hình Web Server trên ServerHTTP}
    \end{figure}
    Chỉnh sửa trang web mặc định để hiển thị thông tin thành viên nhóm.
    \begin{figure}[H]
        \centering
        \includegraphics[width=1\textwidth]{P2_26.png}
        \caption{Chỉnh sửa file index.html trên ServerHTTP}
    \end{figure}
\end{itemize}
Mở PC1, PC2 và PC3, ta thêm địa chỉ IP của ServerDHCP/DNS vào mục DNS Server để các PC có thể phân giải tên miền.
\begin{figure}[H]
    \centering
    \includegraphics[width=1\textwidth]{picture.png}
    \caption{Thêm địa chỉ IP của ServerDHCP/DNS vào mục DNS Server trên PC1}
\end{figure}
Tương tự với PC2 và PC3.
\newpage
\subsubsection{\textbf{Cấu hình DHCP Relay Agent trên Router R4}}
\textbf{Mô tả:} Cấu hình Relay Agent trên cổng GigabitEthernet0/0 để chuyển tiếp gói tin DHCP Discovery sang nhánh Server Farm (172.16.0.34).
\begin{figure}[H]
    \centering
    \includegraphics[width=1\textwidth]{P2_27.png}
    \caption{Mở CLI của R4 và nhập các dòng lệnh để cấu hình DHCP Relay Agent}
\end{figure}
\subsubsection{\textbf{Cấu hình DHCP Client trên các PC trong Office LAN}}
\textbf{Mô tả:} Cấu hình các PC trong Office LAN để nhận địa chỉ IP động từ DHCP Server.
\begin{figure}[H]
    \centering
    \includegraphics[width=1\textwidth]{P2_28.png}
    \caption{Mở PC4, vào Desktop -> IP Configuration, IP Configuration đang để Static}   
\end{figure}
\begin{figure}[H]
    \centering
    \includegraphics[width=1\textwidth]{P2_29.png}
    \caption{Chuyển sang DHCP và chờ một chút để lấy địa chỉ IP động từ DHCP Server}
\end{figure}
\begin{figure}[H]
    \centering
    \includegraphics[width=1\textwidth]{P2_30.png}
    \caption{Địa chỉ IP động đã được cấp cho PC4}
\end{figure}
Làm tương tự để cấu hình DHCP Client cho các PC9, PC5 trong Office LAN.
\subsubsection{\textbf{Kiểm tra Kết nối và Dịch vụ Mạng Toàn Mạng}}
\textbf{Mô tả:} Nhóm tiến hành kiểm tra kết nối mạng và các dịch vụ mạng đã cấu hình trên toàn bộ hệ thống.
\begin{itemize}
    \item \textbf{Kiểm tra kết nối từ PC4 đến ServerHTTP qua tên miền:}
    \begin{figure}[H]
        \centering
        \includegraphics[width=1\textwidth]{P2_32.png}
        \caption{Vào PC4, mở trình duyệt web và truy cập vào \texttt{companyweb.com}.}
    \end{figure}
    \begin{figure}[H]
        \centering
        \includegraphics[width=1\textwidth]{P2_33.png}
        \caption{Trang web hiển thị thông tin thành viên nhóm, chứng tỏ dịch vụ DNS và Web Server hoạt động tốt.}
    \end{figure}
    \item \textbf{Thử ping từ PC4 đến PC1 trong Part 1:}
    \begin{figure}[H]
        \centering
        \includegraphics[width=1\textwidth]{P2_34.png}
        \caption{Mở Command Prompt trên PC4 và ping đến PC1 (192.168.1.10).}
    \end{figure}
\end{itemize}
\newpage
\subsubsection{Tổng kết Phần 2}
\begin{itemize}
    \item Nhóm đã hoàn thành việc mở rộng hệ thống mạng doanh nghiệp bằng cách kết nối chi nhánh mới (Part 2) vào trụ sở chính (Part 1) thông qua đường truyền WAN.
    \item Các dịch vụ mạng thiết yếu như DHCP, DNS và Web Server đã được triển khai và hoạt động ổn định trong toàn bộ hệ thống.
    \item Việc cấu hình DHCP Relay Agent trên Router R4 đã giúp các PC trong Office LAN nhận địa chỉ IP động từ DHCP Server một cách hiệu quả.
    \item Kết nối mạng giữa các thiết bị trong chi nhánh mới và trụ sở chính đã được kiểm tra và xác nhận hoạt động tốt.
    \item Toàn bộ hệ thống mạng đã đáp ứng được các yêu cầu kỹ thuật và mục tiêu đề ra ban đầu.
\end{itemize}
\subsection{Giai đoạn 2: Nâng cấp hệ thống lên VLAN (Advanced Implementation)}
\subsubsection{Chuẩn bị chuyển đổi}
\textbf{Mô tả:} Để chuyển từ mô hình vật lý sang mô hình Router-on-a-Stick, bước đầu tiên nhóm thực hiện là loại bỏ địa chỉ IP trên cổng vật lý và tắt cổng tạm thời để tránh xung đột trong quá trình cấu hình.
Trên Router R4, truy cập interface GigabitEthernet0/0, xóa IP cũ (no ip address) và shutdown cổng.
\begin{figure}[H]
    \centering
    \includegraphics[width=1\textwidth]{P2_35.png}
    \caption{Mở CLI của R4 và nhập các dòng lệnh để xóa cấu hình IP cũ trên cổng vật lý của Router R4.}
\end{figure}
Làm tương tự cho cổng GigabitEthernet0/1 - Switch S5.
\subsubsection{Cấu hình Switch S4}
\begin{enumerate} 
    \item \textbf{Khởi tạo VLAN:} Tạo VLAN 10 với tên OFFICE\_LAN.
    \begin{figure}[H]
        \centering
        \includegraphics[width=1\textwidth]{P2_36.png}
        \caption{Mở CLI của S4 và nhập các dòng lệnh để tạo VLAN 10.}
    \end{figure}
    \item \textbf{Gán cổng Access:} Đưa các cổng kết nối với PC (Range Fa0/1-10) vào VLAN 10.
    \begin{figure}[H]
        \centering
        \includegraphics[width=1\textwidth]{P2_37.png}
        \caption{Cấu hình các cổng người dùng sang chế độ Access VLAN 10.}
    \end{figure}
    \item \textbf{Cấu hình cổng Trunk:} Cấu hình cổng Uplink GigabitEthernet0/1 (nối lên Router R4) sang chế độ Trunk để cho phép VLAN đi qua.
    \begin{figure}[H]
        \centering
        \includegraphics[width=1\textwidth]{P2_38.png}
        \caption{Cấu hình đường Trunk trên cổng Uplink của Switch S4.}
    \end{figure}
    \item \textbf{Kiểm tra cấu hình VLAN và Trunk:} Sử dụng các lệnh show để xác nhận VLAN và trạng thái Trunk.
    \begin{figure}[H]
        \centering
        \includegraphics[width=1\textwidth]{P2_39.png}
        \caption{Mở CLI của S4 và nhập các dòng lệnh để kiểm tra cấu hình VLAN.}
    \end{figure}
    \begin{figure}[H]
        \centering
        \includegraphics[width=1\textwidth]{P2_40.png}
        \caption{Mở CLI của S4 và nhập các dòng lệnh để kiểm tra trạng thái Trunk.}
    \end{figure}
\end{enumerate}
\subsubsection{Cấu hình Switch S5}
\textbf{Mô tả:} Thực hiện các bước tương tự như trên Switch S4 để cấu hình VLAN cho Server Farm.
\begin{enumerate} 
    \item \textbf{Khởi tạo VLAN:} Tạo VLAN 30 với tên SERVER\_FARM.
    \begin{figure}[H]
        \centering
        \includegraphics[width=1\textwidth]{P2_41.png}
        \caption{Mở CLI của S5 và nhập các dòng lệnh để tạo VLAN 30.}
    \end{figure}
    \item \textbf{Gán cổng Access:} Đưa các cổng kết nối với Server (Range Fa0/1-2) vào VLAN 30.
    \begin{figure}[H]
        \centering
        \includegraphics[width=1\textwidth]{P2_42.png}
        \caption{Cấu hình các cổng Server sang chế độ Access VLAN 30.}
    \end{figure}
    \item \textbf{Cấu hình cổng Trunk:} Cấu hình cổng Uplink GigabitEthernet0/1 (nối lên Router R4) sang chế độ Trunk để cho phép VLAN đi qua.
    \begin{figure}[H]
        \centering
        \includegraphics[width=1\textwidth]{P2_43.png}
        \caption{Cấu hình đường Trunk trên cổng Uplink của Switch S5.}
    \end{figure}
\end{enumerate}
\subsubsection{Cấu hình Router R4 cho Router-on-a-Stick và DHCP Relay}
\begin{itemize}
    \item \textbf{Cấu hình cho Office LAN (VLAN 10):} Thao tác chi tiết:
    \begin{enumerate}
        \item Tạo cổng con GigabitEthernet0/0.10.
        \item Đóng gói chuẩn 802.1Q cho VLAN 10 (encapsulation dot1q 10).
        \item Đặt IP Address: 172.16.0.1 255.255.255.224.
        \item Cấu hình Relay Agent: Thêm lệnh ip helper-address 172.16.0.34 để chuyển tiếp yêu cầu DHCP từ VLAN 10 sang Server Farm (VLAN 30).
    \end{enumerate}
    \begin{figure}[H]
        \centering
        \includegraphics[width=1\textwidth]{P2_44.png}
        \caption{Mở CLI của R4 và nhập các dòng lệnh để cấu hình Sub-interface, Encapsulation và IP Helper-Address cho mạng Office LAN (VLAN 10).}
    \end{figure}
    \item \textbf{Cấu hình cho Server Farm (VLAN 30):} Thao tác chi tiết:
    \begin{enumerate}
        \item Tạo cổng con GigabitEthernet0/1.30.
        \item Đóng gói chuẩn 802.1Q cho VLAN 30 (encapsulation dot1q 30).
        \item Đặt IP Address: 172.16.0.33 255.255.255.240.
    \end{enumerate}
    \begin{figure}[H]
        \centering
        \includegraphics[width=1\textwidth]{P2_45.png}
        \caption{Mở CLI của R4 và nhập các dòng lệnh để cấu hình Sub-interface Gateway cho mạng Server Farm (VLAN 30).}        
    \end{figure}
\end{itemize}
\subsubsection{Kiểm tra bảng định tuyến sau chuyển đổi}
\textbf{Mô tả:} Sau khi cấu hình xong các Sub-interface, nhóm thực hiện kiểm tra bảng định tuyến để đảm bảo Router R4 đã nhận diện được các mạng con VLAN dưới dạng Connected (C) và vẫn giữ được đường đi mặc định ra Internet.
\begin{itemize}
    \item Dòng C 172.16.0.0/27: Mạng Office LAN (VLAN 10) đã kết nối.
    \item Dòng C 172.16.0.32/28: Mạng Server Farm (VLAN 30) đã kết nối.
    \item Dòng S* 0.0.0.0/0: Default Route trỏ về R2 vẫn hoạt động tốt.
\end{itemize}
\begin{figure}[H]
    \centering
    \includegraphics[width=1\textwidth]{P2_46.png}
    \caption{Mở CLI của R4 và nhập lệnh \texttt{show ip route} để kiểm tra bảng định tuyến sau khi cấu hình Router-on-a-Stick.}
\end{figure}
\subsubsection{Kiểm tra kết nối và dịch vụ mạng sau khi nâng cấp lên VLAN}
\textbf{Mô tả:} Nhóm tiến hành kiểm tra lại kết nối mạng và các dịch vụ mạng đã cấu hình trên toàn bộ hệ thống sau khi chuyển sang mô hình VLAN.
\begin{itemize}
    \item \textbf{Cấu hình DHCP Client trên các PC trong Office LAN:} Do đã cấu hình DHCP Client từ trước, nhóm chỉ cần kiểm tra lại địa chỉ IP được cấp.
    \begin{figure}[H]
        \centering
        \includegraphics[width=1\textwidth]{P2_50.png}
        \caption{Mở PC4, vào Desktop -> IP Configuration, kiểm tra địa chỉ IP động đã được cấp từ DHCP Server.}
    \end{figure}
    \item \textbf{Kiểm tra kết nối từ PC4 đến ServerHTTP qua tên miền:}
    \begin{figure}[H]
        \centering
        \includegraphics[width=1\textwidth]{P2_47.png}
        \caption{Vào PC4, mở trình duyệt web và truy cập vào \texttt{companyweb.com}.}
    \end{figure}
    \begin{figure}[H]
        \centering
        \includegraphics[width=1\textwidth]{P2_48.png}
        \caption{Trang web hiển thị thông tin thành viên nhóm, chứng tỏ dịch vụ DNS và Web Server hoạt động tốt sau khi nâng cấp lên VLAN.}
    \end{figure}
    \item \textbf{Thử ping từ PC4 đến PC1 trong Part 1:}
    \begin{figure}[H]
        \centering
        \includegraphics[width=1\textwidth]{P2_49.png}
        \caption{Mở Command Prompt trên PC4 và ping đến PC1 (172.16.0.1).}
    \end{figure}
\end{itemize}
\subsubsection{Tổng kết Giai đoạn 2}
\begin{itemize}
    \item Nhóm đã hoàn thành việc nâng cấp hệ thống mạng doanh nghiệp lên mô hình Router-on-a-Stick với VLAN, giúp tối ưu hóa quản lý mạng và phân tách lưu lượng.
    \item Các dịch vụ mạng thiết yếu như DHCP, DNS và Web Server đã được kiểm tra và xác nhận hoạt động ổn định trong mô hình VLAN.
    \item Việc cấu hình DHCP Relay Agent trên Router R4 đã giúp các PC trong Office LAN nhận địa chỉ IP động từ DHCP Server một cách hiệu quả trong môi trường VLAN.
    \item Kết nối mạng giữa các thiết bị trong chi nhánh mới và trụ sở chính đã được kiểm tra và xác nhận hoạt động tốt sau khi nâng cấp.
\end{itemize}
\begin{figure}[H]
    \centering
    \includegraphics[width=1\textwidth]{P2_51.png}
    \caption{Hình ảnh tổng quan mô hình mạng sau khi hoàn thành Giai đoạn 2 với VLAN.}
\end{figure}
\newpage
\section{KẾT LUẬN VÀ HƯỚNG PHÁT TRIỂN}

\subsection{Kết luận}
Nhóm đã hoàn thành việc kết nối thông suốt giữa hai phân vùng mạng khác nhau. Các dịch vụ DHCP, DNS và Web hoạt động ổn định, PC ở Phần 1 đã có thể truy cập trang web ở Phần 2 thông qua tên miền.

\subsection{Hạn chế}
Sử dụng giao thức định tuyến RIPv2 có tốc độ hội tụ chậm và tốn băng thông do gửi bảng định tuyến theo chu kỳ.

\subsection{Hướng phát triển}
Nâng cấp giao thức định tuyến lên OSPF hoặc EIGRP để tối ưu hóa hiệu năng cho mạng lớn hơn, triển khai thêm Firewall để bảo mật hệ thống Server.

\end{document}